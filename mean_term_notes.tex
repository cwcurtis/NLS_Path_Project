 \documentclass[a4paper,11pt]{article}

\usepackage{amsmath}
\usepackage{amssymb}
\usepackage{amsthm}
\usepackage{graphicx}
\usepackage{caption}
\usepackage{subcaption}

\newtheorem{thm}{Theorem}
\newtheorem{lem}{Lemma}

\newcommand{\beq}{\begin{equation}}
\newcommand{\eeq}{\end{equation}}

\newcommand{\ba}{\begin{array}}
\newcommand{\ea}{\end{array}}

\newcommand{\bea}{\begin{eqnarray}}
\newcommand{\eea}{\end{eqnarray}}

\newcommand{\bc}{\begin{center}}
\newcommand{\ec}{\end{center}}

\newcommand{\ds}{\displaystyle}

\newcommand{\bt}{\begin{tabular}}
\newcommand{\et}{\end{tabular}}

\newcommand{\bi}{\begin{itemize}}
\newcommand{\ei}{\end{itemize}}

\newcommand{\bd}{\begin{description}}
\newcommand{\ed}{\end{description}}

\newcommand{\bp}{\begin{pmatrix}}
\newcommand{\ep}{\end{pmatrix}}

\newcommand{\p}{\partial}
\newcommand{\sech}{\mbox{sech}}

\newcommand{\cf}{{\it cf.}~}

\newcommand{\ltwo}{L_{2}(\mathbb{R}^{2})}
\newcommand{\smooth}{C^{\infty}_{0}(\mathbb{R}^{2})}

\newcommand{\br}{{\bf r}}
\newcommand{\bk}{{\bf k}}
\newcommand{\bv}{{\bf v}}

\newcommand{\gnorm}[1]{\left|\left| #1\right|\right|}
\newcommand{\ipro}[2]{\left<#1,#2 \right>}
\title{Notes}
\date{}

\begin{document}
\maketitle
\subsection*{Issues with Mean Terms}

From John's writeup, we have that 
\[
\phi(x,z,t) = \epsilon^{2}\bar{\phi}_{0}(X,Z,T) + \cdots,
\]
where 
\begin{align*}
\p_{X}^{2}\bar{\phi}_{0} + \p_{Z}^{2}\bar{\phi}_{0} = 0, & ~-\infty < Z < 0 \\
\p_{Z}\bar{\phi}_{0} = 0, & ~Z = -\infty\\
\left.\p_{Z}\phi_{0}\right|_{Z=0} = 2\omega_{0}\p_{X}\left|B\right|^{2}, & ~ Z=0.
\end{align*}
Using Fourier transforms, we can then readily show that 
\[
\bar{\phi}_{0}(X,Z,T) = 2\omega_{0}\mathcal{H}\left(\frac{1}{2\pi}\int_{\mathbb{R}} \widehat{\left| B\right|^{2}}(k,T) e^{ikX}e^{|k|Z} dk \right),
\]
where $\mathcal{H}$ is the Hilbert transform.  If we then let, sticking with John's scalings,
\begin{align*}
q(x,t) = & \phi(x,\eta,t) \\
= & \epsilon^{2}\bar{\phi}_{0}(X,\epsilon \eta,T) + \cdots,
\end{align*}
then if we define $Q = q_{x}$, we see this gives us 
\[
Q = \epsilon^{3}\bar{\phi}_{0,X}(X,0,T) + \cdots
\]
Note, this explains why in my write-up, $\eta_{0}$ and $q_{0}$ appear at the same order.  Likewise, we now see that the mean term associated with $Q$ is given by 
\[
Q = \epsilon^{3}\left( 2\omega_{0}\mathcal{H}\p_{X}\left|B\right|^{2}\right)
\]
If you take into account that I use $e^{ik_{0}+i\Omega t}$ instead of $e^{ik_{0}-i\omega_{0}t}$ as in John's writeup, and if you note that I have my surface be at $z=\epsilon \eta$, so that all of my terms are an order of $\epsilon$ smaller than John's, then in essence, we are getting the same result.  The only quirk is that in my formula for $q_{0}$, if $\omega =0$, I have 
\[
q_{0} = -2\mbox{sgn}(k_{0})\Omega \mathcal{H}\p_{\xi}\left|\eta_{1} \right|^{2}.
\]
Thus, the presence of the $\mbox{sgn}(k_{0})$ makes my results somewhat different from John's, though that may have to do with some of the analytical oddities we noted with regards to the traditional approach used in deriving the NLS equation.  

\subsection*{Issues with $\Omega$}
If to leading order we use 
\[
Q \sim q_{1}e^{ik_{0}x + i\Omega t}, ~ \eta \sim \eta_{1}e^{ik_{0}x + i\Omega t}
\]
we get $q_{1}=-\mbox{sgn}(k_{0})\Omega \eta_{1}$ and the characteristic equation 
\[
\Omega^{2} -s \omega \Omega - |k_{0}|(1+\tilde{\sigma}k_{0}^{2}) = 0, ~ s = \mbox{sgn}(k_{0}),
\]
with roots 
\[
\Omega_{\pm} = \frac{1}{2}\left(s\omega \pm \sqrt{\omega^{2} + 4|k_{0}|(1+\tilde{\sigma}k_{0}^{2})} \right).
\]
Had we used 
\[
Q \sim q_{1}e^{ik_{0}x - i\Omega t}, ~ \eta \sim \eta_{1}e^{ik_{0}x - i\Omega t}
\]
we get $q_{1}=s\Omega \eta_{1}$ and the characteristic equation 
\[
\Omega^{2} + s \omega \Omega - |k_{0}|(1+\tilde{\sigma}k_{0}^{2}) = 0, ~ s = \mbox{sgn}(k_{0}),
\]
with roots 
\[
\Omega_{\pm} = \frac{1}{2}\left(-s\omega \pm \sqrt{\omega^{2} + 4|k_{0}|(1+\tilde{\sigma}k_{0}^{2})} \right).
\]
As we can see, one dispersion relationship is just the negative of the other, though we see the pairings are a bit subtle in the sense that we should think in terms of the pairs
\[
\frac{1}{2}\left(s\omega + \sqrt{\omega^{2} + 4|k_{0}|(1+\tilde{\sigma}k_{0}^{2})} \right), ~ \frac{1}{2}\left(-s\omega - \sqrt{\omega^{2} + 4|k_{0}|(1+\tilde{\sigma}k_{0}^{2})} \right)
\]
and
\[
\frac{1}{2}\left(s\omega - \sqrt{\omega^{2} + 4|k_{0}|(1+\tilde{\sigma}k_{0}^{2})} \right), ~ \frac{1}{2}\left(-s\omega + \sqrt{\omega^{2} + 4|k_{0}|(1+\tilde{\sigma}k_{0}^{2})} \right).
\]

Likewise, moving on to the question of the group velocity, using the $e^{ik_{0}x+i\Omega t}$ convention, this comes down to solving the system of equations
\[
is q_{1} + i\Omega \eta_{1} + \epsilon \p_{T}\eta_{1} \sim 0
\]
and
\[
i\Omega q_{1} + i\omega \Omega \eta_{1} + ik_{0}(1+\tilde{\sigma}k_{0}^{2})\eta_{1} + \epsilon\left(\p_{T}q_{1} + \omega\p_{T}\eta_{1} + (1+3\tilde{\sigma}k_{0}^{2})\p_{X}\eta_{1} \right) \sim 0.
\]
Note these are all $e^{i\theta}$ terms, and thus at this order can only come from linear contributions.  Doing the usual thing then ultimately gets us 
\[
(1+3\tilde{\sigma}k_{0}^{2})\p_{X}\eta_{1} + (\omega-2s\Omega)\p_{T}\eta_{1} = 0, 
\]
which then, letting $\xi = X + c_{g}T$, gives 
\[
c_{g} = \frac{1+3\tilde{\sigma}k_{0}^{2}}{2s\Omega - \omega}.
\]

Taking the other convention, we get 
\[
is q_{1} - i\Omega \eta_{1} + \epsilon \p_{T}\eta_{1} \sim 0
\]
and
\[
-i\Omega q_{1} - i\omega \Omega \eta_{1} + ik_{0}(1+\tilde{\sigma}k_{0}^{2})\eta_{1} + \epsilon\left(\p_{T}q_{1} + \omega\p_{T}\eta_{1} + (1+3\tilde{\sigma}k_{0}^{2})\p_{X}\eta_{1} \right) \sim 0.
\]
This gives
\[
(1+3\tilde{\sigma}k_{0}^{2})\p_{X}\eta_{1} + (\omega+2s\Omega)\p_{T}\eta_{1} = 0.
\]
So again, we are only a sign difference away from the other convention.  Now, you would usually let your traveling coordinate be $\xi = X - c_{g}T$, so that 
\[
c_{g} = \frac{1+3\tilde{\sigma}k_{0}^{2}}{2s\Omega + \omega}.
\]
Choosing $\xi=X+c_{g}T$ instead though gives us
\[
c_{g} = \frac{1+3\tilde{\sigma}k_{0}^{2}}{-2s\Omega - \omega},
\]
 which I guess looks more like the negative of the case above.  Suffice to say, the freedom of choice we have in making these various choices allows for quite a degree of variability in where various signs appear.  
\subsection*{Deriving the Nonlinear Term}
So, because we have clearly committed many sins in a previous life, we are obliged to expand 
\begin{align*}
Q_{t} + \omega \eta_{t} + \eta_{x} - \tilde{\sigma}\eta_{xxx} + \frac{\epsilon}{2}\p_{x}\left(-\eta_{t}^{2} + (Q+\omega \eta)^{2} \right) \\
+ \epsilon^{2}\p_{x}\left( \frac{3}{2}\tilde{\sigma}\eta_{x}^{2}\eta_{xx} - \eta_{t}\eta_{x} \left(Q  + \omega \eta \right) \right) = 0, 
\end{align*}
and
\begin{multline*}
\eta_{t} + \mathcal{H}Q + \epsilon\p_{x}\left(Q\eta + \frac{\omega}{2}\eta^{2}-\frac{1}{2}\mathcal{H}\p_{t}\eta^{2} \right)\\ 
- \epsilon^{2}\p_{x}^{2}\left(\frac{1}{6}\p_{t}\eta^{3} - \frac{\omega}{3}\mathcal{H}\eta^{3} + \frac{1}{2}\mathcal{H}\left(\eta^{2} Q\right)  \right) = 0.
\end{multline*}
Given that $\mathcal{H}^{2}=-1$, we note that the second equation in effect gives us $Q$ in terms of $\eta$ since 
\begin{multline*}
Q = \mathcal{H}\eta_{t} + \epsilon\p_{x}\mathcal{H}\left(Q\eta + \frac{\omega}{2}\eta^{2}-\frac{1}{2}\mathcal{H}\p_{t}\eta^{2} \right)\\ 
- \epsilon^{2}\p_{x}^{2}\mathcal{H}\left(\frac{1}{6}\p_{t}\eta^{3} - \frac{\omega}{3}\mathcal{H}\eta^{3} + \frac{1}{2}\mathcal{H}\left(\eta^{2} Q\right)  \right).
\end{multline*}
Using this, we can readily show that 
\[
Q = \mathcal{H}\eta_{t} + \epsilon R_{1} + \epsilon^{2}R_{2} + \cdots,
\]
where
\begin{align*}
R_{1} = & \p_{x}\mathcal{H}\left(\eta\mathcal{H}\eta_{t} + \frac{\omega}{2}\eta^{2}-\frac{1}{2}\mathcal{H}\p_{t}\eta^{2} \right)\\
R_{2} = & \p_{x}\mathcal{H}\left(\eta R_{1} \right) - \p_{x}^{2}\mathcal{H}\left(\frac{1}{6}\p_{t}\eta^{3} - \frac{\omega}{3}\mathcal{H}\eta^{3} + \frac{1}{2}\mathcal{H}\left(\eta^{2} \mathcal{H}\eta_{t}\right)\right).
\end{align*}
Having ``solved'' for $Q$, we then substitute this result into the Bernoulli equation, thereby giving us a closed system in terms of $\eta$.  As an aside, we see that this approach would greatly simplify most attempts at deriving higher order corrections to the NLS equation.  

Using the expansion
\[
\eta(x,t) = \tilde{\eta}(x,X,t,T,\tau) + \mathcal{O}(\epsilon^{3}),
\]
where, letting $X=\epsilon x$, $T=\epsilon t$, $\tau = \epsilon^{2}t$, and $\xi = X + c_{g}T$, 
\begin{multline}
\tilde{\eta}(x,X,t,T,\tau) = \epsilon^{2}\eta_{0}(\xi,\tau) + \eta_{1}(\xi,\tau)e^{i\theta(x,t)} \\
+ \epsilon \eta_{2}(\xi,\tau)e^{2i\theta(x,t)}  + \epsilon^{2}\eta_{3}(\xi,\tau)e^{3i\theta(x,t)} + \mbox{c.c.},
\label{nlssurfexpan}
\end{multline}
then using SAGE and patience, we get 
\[
Q = \epsilon^{2}q_{0} + q_{1}e^{i\theta} + \epsilon q_{2}e^{2i\theta} + \epsilon^{2}q_{3}e^{3i\theta} + \cdots
\]
where
\[
q_{0} = (\omega - 2s\Omega)\p_{\xi}\mathcal{H}\left|\eta_{1}\right|^{2}
\]
\begin{multline*}
q_{1} = -s\Omega \eta_{1} + \epsilon ic_{g}s\p_{\xi}\eta_{1} \\
+ \epsilon^{2}\left(  is\p_{\tau}\eta_{1} + k_{0}^{2}\left(\omega-s\frac{\Omega}{2} \right)\left|\eta_{1}\right|^{2}\eta_{1} + k_{0}(2\Omega-s\omega)\eta_{1}^{\ast}\eta_{2}\right) 
\end{multline*}
and
\[
q_{2} = -\left|k_0\right| \omega n_1^2  - 2s\Omega n_2 + i\epsilon(s c_g \p_{\xi}n_2 + \omega n_1 \p_{\xi}n_1 )
\]
\end{document}